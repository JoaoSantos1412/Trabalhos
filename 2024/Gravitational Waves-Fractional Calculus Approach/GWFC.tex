\RequirePackage{snapshot}
\documentclass[a4paper,12pt]{extarticle}
\usepackage{amsthm}
\newtheorem{definition}{Definition}[section]
\newtheorem{Lemma}{Lemma}[section]
\usepackage[english]{babel}
\usepackage{multirow}
\usepackage{graphicx}
\usepackage{bm}
\usepackage{blindtext}
\usepackage[a4paper]{geometry}
\usepackage{fancyhdr}
\usepackage{pgfplots}
\usepackage{pgfplotstable}
\usepackage{tocloft}
\usepackage{amsfonts}
\usepackage{hyperref}
\usepackage{placeins}
\usepackage[lastpage,user]{zref}
\usepackage{adjustbox}
\hypersetup{hidelinks}
\usepackage{nicematrix}
\usepackage{tabularx}
\usepackage{amsmath, nccmath}
\usepackage{titlesec}
\usepackage{float, caption, booktabs, cellspace}
\NiceMatrixOptions{cell-space-top-limit=5pt,cell-space-bottom-limit=5pt}
\setlength\heavyrulewidth{0.4pt} % espessura de top and bottom %
\setlength\lightrulewidth{0.4pt} % espessura de mid %
\newcommand{\mn}{{\mu\nu}}
\usepackage[a4paper]{geometry}
%\geometry{showframe}
\newcommand{\diag}[1]{\text{diag}(#1)}
\numberwithin{equation}{subsection}
\newcommand{\sign}{\text{sign}(x)}
\usepackage{float}
\usepackage{amsmath}
\usepackage{moresize}
\usepackage{extsizes}
%\usepackage[portuguese]{babel}
\usepackage{multirow}
\usepackage{graphicx}
\usepackage{geometry}
\usepackage{fancyhdr}
\usepackage{tikz}
\usepackage{adjustbox}
\usepackage{pgfplots}
\usepackage{pdfpages}
\usepackage{pgfplotstable}
\usepackage{tocloft}
\usepackage{hyperref}
\usepackage{gensymb}
\usepackage{mathtools}
\usepackage{placeins}
\hypersetup{hidelinks}
\usepackage{nicematrix}
\usepackage{tabularx}
\usepackage{float, caption, booktabs, cellspace}
\begin{document}
\begin{center}
{\LARGE \bf Gravitational Waves \\[.6ex] 
\large a Fractional Calculus Approach}
\vspace{8mm}

{\large \bf João Miguel Costa Santos}
\vspace{3mm}


BSc student at Universidade da Beira Interior, Covilhã, Portugal \\
e-mail: \url{joao.miguel.costa.santos@ubi.pt}
\vspace{2mm}


\end{center}
\vspace{10mm}
\noindent
{\bf Abstract:} This paper explores the application of fractional calculus to the study of gravitational waves, providing a novel approach to understanding their behavior. We start with a detailed overview of the fundamental concepts of General Relativity, including the spacetime notation, metric tensor, and the Einstein equations. We then introduce the principles of fractional calculus, focusing on integrals and derivatives, and discuss their relevance to physical phenomena. The core of our work presents a fractional calculus perspective on gravitational waves, emphasizing the use of Riemann-Liouville, Caputo, and Riesz fractional derivatives, as well as the fractional Laplacian.\\[.5em]
{\bf Keywords:} Gravitational Waves, Fractional Calculus,  Riemann-Liouville fractional derivatives,  Caputo fractional derivatives,  Riesz fractional derivatives, fractional Laplacian.\\
\vspace{4mm}
\pagenumbering{gobble}
\tableofcontents
\clearpage
\newpage
\pagestyle{headings}
\setcounter{page}{1}
\pagenumbering{arabic}
\section{Notation Conventions}
Let us introduce two pivotal concepts within the domain of general relativity: the notation conventions and the metric utilized throughout this paper.
\subsection{Spacetime Notation}
Minkowski, in order to simplify special relativity, united the three dimensions of space and the one dimension of time into the so called 4-dimensional spacetime, whose coordinates are denoted as:
\begin{equation}
\label{eqn:STC}
(x^0,x^1,x^2,x^3)\equiv(ct,x,y,z)
\end{equation}
where the superscripts are indices (and not exponents!). The time component $(ct)$ is adjusted by the speed of light to ensure consistency in units with the other coordinates. This is also commonly written in $c=1$ units, and therefore
\begin{equation}
(x^0,x^1,x^2,x^3)\equiv(t,x,y,z)
\end{equation}
\subsection{Metric Tensor}
\begin{equation}
\begin{aligned}
ds^2=\sum_{\mu\nu}\eta_{\mu\nu}dx^\mu dx^\nu&=-(dx^0)^2+(dx^1)^2+(dx^2)^2+(dx^3)^2,\\
&=-c^2dt^2+dx^2+dy^2+dz^2.
\end{aligned}
\end{equation}
The quantity $\eta_\mn$ is represented by the following diagonal matrix
\begin{equation}
\label{eqn:diag}
\eta_{\mu\nu}=
\begin{pmatrix}
-1 & 0 & 0 & 0\\
0 & 1 &0 & 0\\
0 & 0 & 1 & 0\\
0 & 0 & 0 & 1
\end{pmatrix}
=\diag{-1,1,1,1}
\end{equation}
and is referred to as the metric tensor of Minkowski space. (a lot of books choose the symmetric metric $\eta_\mn=\diag{1,-1,-1,-1}$, however the one mentioned in eq.  \eqref{eqn:diag} is the most used in relativity.)
\newpage
\section{Preliminary Concepts of GR}
Before we start, due to the inherent complexity of gravitational waves, it is imperative to provide thorough mathematical analysis beforehand to ensure a comprehensive understanding.
\subsection{Euclidean Geometry}
\subsubsection{Parameterizing in Different Coordinate Systems}
Suppose we have a 3-dimensional euclidean space with a cartesian coordinate system $(x, y, z)$, and a set of mutually orthogonal unit vectors $(i,j,k)$. Now, let's assume another coordinate system $(u,v,w)$ that is related to the $(x,y,z)$ coordinates by transformations
\begin{align}
x=x(u,v,w) && y=y(u,v,w) && z=z(u,v,w)
\end{align}
that are invertible so that its possible to solve for $(u,v,w)$ in term of $(x,y,z)$. From these coordinates we can create a position vector $\bm r$ for any point in terms of the $(u,v,w)$ coordinates:
 \begin{equation}
\bm r = x(u,v,w) \bm i + y(u,v,w) \bm j + z(u,v,w) \bm k.
\end{equation}
\subsubsection{Basis Vectors}
Tangent basis
\begin{align}
\label{eqn:basis_i}
\bm e_u\equiv \frac{\partial\bm r}{\partial u} && \bm e_v\equiv \frac{\partial\bm r}{\partial v} && \bm e_w\equiv \frac{\partial\bm r}{\partial w}
\end{align}
Dual basis
\begin{equation}
\label{eqn:basis^i}
\begin{aligned}
\bm e^u \equiv \bm \nabla u &= \frac{\partial u}{\partial x} \bm i + \frac{\partial u}{\partial y} \bm j + \frac{\partial u}{\partial z} \bm k\\ 
\bm e^v \equiv \bm \nabla v &= \frac{\partial v}{\partial x} \bm i + \frac{\partial v}{\partial y} \bm j + \frac{\partial v}{\partial z} \bm k\\
\bm e^w \equiv \bm \nabla u &= \frac{\partial w}{\partial x} \bm i + \frac{\partial w}{\partial y} \bm j + \frac{\partial w}{\partial z} \bm k
\end{aligned}
\end{equation}
\subsubsection{Expansion of vectors and dual vectors}
We can express an arbitrary vector $\bm V$ in terms of the tangent basis $(\bm e_i)$, and similarly, an arbitrary dual vector $\bm \omega$ in terms of the dual basis $(\bm e^i)$:
\begin{align}
\label{eqn:v}
&\bm V = V^1\bm e_1 + V^2\bm e_2 + V^3\bm e_3=\sum_i V^i\bm e_i\equiv V^i\bm e_i,\\
&\bm \omega = \omega_1\bm e^1 + \omega_2\bm e^2 + \omega_3\bm e^3=\sum_i \omega_i\bm e^i\equiv \omega_i\bm e^i.
\end{align}
where the upper-index coefficients $V^i$ are referred to as the components of the vector in the basis $\bm e_i$ and the lower-index coefficients $\omega_i$ are referred to as the components of the dual vector in the basis $\bm e^i$. These components, as they are in different basis, usually are distinct, nevertheless, there is a relation between their spaces.
\subsubsection{Vector Scalar Product and the Metric Tensor}
The scalar product of two vector can be written as follows
\begin{equation}
\label{eqn:PE1}
\bm A \cdot \bm B=(A^i\bm e_i)\cdot(B^j\bm e_j)=\bm e_i \cdot\bm e_j A^iB^j=g_{ij}A^iB^j
\end{equation}
where $g_{ij}$ are the metric tensor components (in this basis) and are defined by
\begin{equation}
g_{ij}\equiv\bm e_i \cdot\bm e_j
\end{equation}
The same can be done to the scalar product of two dual vectors:
\begin{equation}
\label{eqn:PE2}
\bm\alpha\cdot\bm\beta=(\alpha_i\bm e^i)\cdot(\beta_j\bm e^j)=\bm e^i \cdot\bm e^j\alpha_i\beta_j=g^{ij}\alpha_i\beta_j
\end{equation}
where the metric tensor components (in this basis) $g^{ij}$ are defined by
\begin{equation}
g^{ij}\equiv\bm e^i \cdot\bm e^j
\end{equation}
It is also possible to have the scalar product between a dual vector and a vector (and vice versa)
\begin{equation}
\label{eqn:PE3}
\bm\alpha\cdot\bm B=(\alpha_i\bm e^i)\cdot(B^j\bm e_j)=\bm e^i \cdot\bm e_j \alpha_iB^j=g^i_j\alpha_iB^j
\end{equation}
where the metric tensor $g^i_j$ is defined by
\begin{equation}
g^i_j\equiv\bm e^i \cdot\bm e_j
\end{equation}
The basis $\bm e^i$ and $\bm e_i$ are defined in such a way that:
\begin{equation}
\bm e^i\cdot\bm e_j=\delta^i_j=
\begin{cases}
1 \text{ if } i=j\\
0 \text { if } i\neq j
\end{cases}
\end{equation}
\subsubsection{Relationship of Vectors and Dual Vectors}
The scalar product of two arbitrary vectors $\bm A$ and $\bm B$ can be written in three different ways, as seen in \eqref{eqn:PE1}, \eqref{eqn:PE2} and \eqref{eqn:PE3}:
\begin{equation}
\bm A \cdot \bm B=
\begin{cases}
(A^i\bm e_i)\cdot(B^j\bm e_j)=g_{ij}A^iB^j\\
(A_i\bm e^i)\cdot(B_ j\bm e^j)=g^{ij}A_iB_j\\
(A^i\bm e_i)\cdot(B_j\bm e^j)=g_{i}^jA^iB_j=A^iB_i=A_iB^i
\end{cases}
\end{equation}
Thus, $A_iB^i=g_{ij}A^iB^j$, which means the vector $\bm B$ is arbitrary and therefore
\begin{equation}
A_i=g_{ij}A^j
\end{equation}
Likewise, $A^iB_i=g^{ij}A_iB^j$, which means the vector $\bm B$ is arbitrary and therefore
\begin{equation}
A^i=g^{ij}A_j
\end{equation}
This means a contravariant component can be written in terms of the covariant components and vice versa as follows:
\begin{align}
A_i=\bm A\cdot \bm e_i = A^j\bm e_j\cdot \bm e_i=g_{ij}A^j\\
A^i=\bm A\cdot \bm e^i = A_j\bm e^j\cdot \bm e^i=g^{ij}A_j
\end{align}
\subsubsection{Properties of the Metric Tensor}
The metric tensor is defined by the scalar products of basis vectors, which inherently results in its symmetry with respect to its indices:
\begin{align}
g^{ij}=g^{ji} && g_{ij}=g_{ji}
\end{align}
and, as seen before
\begin{align}
\label{eqn:a}
A_i=g_{ij}A^j && A^i=g^{ij}A_j
\end{align}
From the equations in \eqref{eqn:a}, $A^i=g_{ij}A^j=g^{ij}g_{jk}A^k$. Thus, the metric tensor obeys
\begin{equation}
g^{ij}g_{jk}=g_{kj}g^{ji}=\delta^i_k
\end{equation}
\subsubsection{Differentiation}
Using a vector $\bm V$ as defined in \eqref{eqn:v} and taking its first derivative
\begin{equation}
\frac{\partial \bm V}{\partial x^j}=\frac{\partial V_i}{\partial x^j}\bm e_i+V^i\frac{\partial \bm e_i}{\partial x^j}
\end{equation}
The first term is the change in the component $V^i$ and the second term represents the change in the basis vectors $\bm e_i$. This second term is zero for a basis independent of coordinates. However, if the basis depends on the coordinates, its usually not zero. The second term is also a vector and can be represented as
\begin{equation}
\frac{\partial \bm e_i}{\partial x^j}=\Gamma^k_{ij}\bm e_k
\end{equation}
and these are called Christoffel symbols.
%-------------------------------------------------------%
\subsection{Non Euclidean Geometry}
The previous section provided a mathematical foundation for Euclidean spaces ranging from 1D to 3D. However, as mentioned earlier, general relativity employs 4-vectors. Therefore, the following section will be similar to the previous one but will focus on 4-dimensional space (4-vectors).
\subsubsection{Spacetime Coordinates}
As seen in \eqref{eqn:STC},
\begin{equation}
x\equiv x^\mu=(x^0,x^1,x^2,x^3)=(ct,\bm x)
\end{equation}
where $\bm x$ denotes the spacial components $(x^1,x^2,x^3)$. The first coordinate ($x^0$) is called timelike and the three last components $(\bm x)$ are called spacelike. The placement of the $\mu$ index is also important as there will be a difference in the sign of the 4-vector. (\emph{Note: the bold vector $\bm x$ is a normal vector that we are used to, whereas the non-bold “vector” $x$ is the 4-vector.})
\subsubsection{Coordinate and Non-coordinate Bases}
The basis vectors can be written as
\begin{equation}
e_\mu=\frac{\partial}{\partial x^\mu}=\partial_\mu
\end{equation}
wich then allows an arbitrary vector to be expanded as
\begin{equation}
V=V^\mu e_\mu=V^\mu\partial_\mu
\end{equation}
The distance between two nearby points is $ds=e_\mu(x) dx^\mu$, thus
\begin{equation}
ds^2=ds\cdot ds=(e_\mu\cdot e_\nu)dx^\mu dx^\nu=g_\mn dx^\mu dx^\nu
\end{equation}
where the metric tensor $g_\mn$ is defined by
\begin{equation}
g_\mn(x)=e_\mu(x)\cdot e_\nu(x)
\end{equation}
The scalar product of two vector can then be written as
\begin{equation}
A\cdot B=(A^\mu e_\mu)\cdot(B^\nu e_\nu)=g_\mn A^\mu B^\nu
\end{equation}
Likewise, a set of dual basis vectors ($e^\mu$) may be used to expand dual vectors $\omega$ as
\begin{equation}
\omega=\omega_\mu e^\mu
\end{equation}
and the scalar of two dual vectors
\begin{equation}
\alpha\cdot\beta=g^\mn\alpha_\mu\beta_\nu
\end{equation}
we can also calculate the scalar product between a vector and a dual vector
\begin{equation}
\omega(V)=\omega_\mu e^\mu(V^\nu e_\nu)=g_\nu^\mu\omega_\mu V^\nu =\omega_\mu V^\mu
\end{equation}
where
\begin{equation}
g^\nu_\mu=\delta_\nu^\mu=
\begin{cases}
1\quad\mu=\nu\\
0\quad\mu\neq\nu
\end{cases}
\end{equation}
\subsubsection{Tensors of Higher Rank}
Tensor products of lower-rank tensors, represented by $\otimes$, can be used to create higher-rank tensors. Tensor products of vector and dual vector spaces can be used to build a basis for these higher-rank tensors. The following is an expression for a mixed tensor $T$ of rank $(p,q)$ schematically:
\begin{equation}
T=T^{\mu_1\mu_2...\mu_p}_{\nu_1\nu_2...\nu_q}e_{\mu_1}\otimes e_{\mu_2} \otimes ... \otimes e_{\mu_p} \otimes e^{\nu_1}\otimes e^{\nu_2} \otimes ... \otimes e^{\nu_p}
\end{equation}
where $\{e_\mu\}$ is a vector basis and ${e^\nu}$ is a dual vector basis. A rank-2 tensor $T$ covariant, contravariant and mixed components are given by
\begin{align}
T(e_\mu,e_\nu)=T_\mn && T(e^\mu,e^\nu)=T^\mn && T(e_\mu,e^\nu)=T_\mu^\nu && T(e^\mu,e_\nu)=T^\mu_\nu 
\end{align}
thus, the vectors $A$ and $B$,
\begin{equation}
T(A,B)=T(A^\mu e_\mu,B^\nu e_\nu)=T(e_\mu,e_\nu)A^\mu B^\nu=T_\mn A^\mu B^\nu
\end{equation}
Finnaly, lets consider the scalar product $A\cdot B=g_\mn A^\mu B^\nu$
\begin{equation}
\begin{aligned}
g(A,B)&=g(A^\mu e_\mu,B^\nu e_\nu)=g(e_\mu,e_\nu)A^\mu B^\nu\\
&=g_\mn A^\mu B^\nu=A^\mu B_\mu\, \in \, \mathbb{R}
\end{aligned}
\end{equation}
Thus, we conclude that $g_\mn$ represents a rank-2 tensor of type (0,2).

\subsubsection{Identification of vectors and dual vectors}
Considering the metric tensor mentioned in the previous subsection, if we have a vector $V$, we can write a dual vector of the type $\tilde V\equiv g(V,\cdot)$:
\begin{equation}
\begin{aligned}
V_\mu&\equiv \tilde V(e_\mu)=g(V,e_\mu)\\
&=g(V^\nu e_\nu,e_mu)\\
&=V^\nu g(e_\nu,e_\mu)\\
&=g_\mn V^\nu
\end{aligned}
\end{equation}
Likewise, since $g_\mn$ and $g^\mn$ are matrix inverses, $V^\mu=g^\mn V_\nu$ \emph{(reminder: this is an implicit sum)}. In summary, we can write
\begin{align}
V_\mu=g_\mn V^\nu && V^\mu=g^\mn V_nu
\end{align}
The same properties can be used to raise or lower any index for a tensor of any rank:
\begin{align}
A^\mn=g^{\mu\alpha}g^{\nu\beta}A_{\alpha\beta} && A_{\mu\nu\lambda\sigma}=g_{\mu\rho}A^\rho_{\nu\lambda\sigma}
\end{align} 

\subsubsection{Transformation laws}
A brief and condensed table of the transformation laws of Tensors is
\begin{table}[H]
\centering
\begin{NiceTabular}{ll}
\toprule
\text{Tensor} & \text{Transformation Law}\\ \midrule
\text{Scalar} & $\phi'=\phi$\\
\text{Dual vector} & $A^{'}_\mu=\frac{\partial x^\nu}{\partial x^{'\mu}}A_\nu$\\
\text{Vector} & $A^{'\mu}=\frac{\partial x^{'\mu}}{\partial x^\nu}A^\nu$\\
\text{Covariant rank-2} & $T^{'}_\mn=\frac{\partial x^\alpha}{\partial x^{'\mu}}\frac{\partial x^\beta}{\partial x^{'\nu}}T_{\alpha\beta}$\\
\text{Contravariant rank-2} & $T^{'\mn}=\frac{\partial x^{'\mu}}{\partial x^\alpha}\frac{\partial x^{'\nu}}{\partial x^\beta}T^{\alpha\beta}$\\
\text{Mixed rank-2} & $T^{'\nu}_\mu=\frac{\partial x^\alpha}{\partial x^{'\mu}}\frac{\partial x^{'\nu}}{\partial x^\beta}T^\beta_\alpha$\\ \bottomrule
\end{NiceTabular}
\end{table}

%-------------------------------------------------------%
%-------------------------------------------------------%
%-------------------------------------------------------%
%-------------------------------------------------------%
%-------------------------------------------------------%
%-------------------------------------------------------%
%-------------------------------------------------------%
%-------------------------------------------------------%
\newpage
\subsection{The Einstein Equations}
Having covered the fundamentals of Non Euclidian geometry, which is crucial to General Relativity, we may now finally discuss the Einstein Equations.\\ \\
The contraction of the Riemann tensor with the metric tensor yields the symmetric Ricci tensor $R_\mn$: 
\begin{align}
R_{\mu\nu}&=R_{\nu\mu}=g^{\lambda\sigma}R_{\lambda\mu\sigma\nu}=R^\sigma_{\mu\sigma\nu}\\
&=\Gamma^\lambda_{\mu\nu,\lambda}-\Gamma^{\lambda}_{\mu\lambda,\nu}+\Gamma_{\mu\nu}^\lambda\Gamma^\sigma_{\lambda\sigma}-\Gamma_{\mu\nu}^\sigma\Gamma^\lambda_{\nu\sigma}
\end{align}
and so, $R$, the Ricci scalar, is simply:
\begin{equation}
R=R_\mu^\mu=g^{\mu\nu}R_{\mu\nu}
\end{equation}
The Riemann tensor obeys the Bianchi identity
\begin{equation}
\nabla_\lambda R_{\mu\nu\alpha\beta}+\nabla_\beta R_{\mu\nu\lambda\alpha}+\nabla_\alpha R_{\mu\nu\beta\lambda}
\end{equation}
We obtain the following important identity after a few more contractions of the Bianchi identity with the metric tensor.
\begin{equation}
\lambda_\mu G^{\mu\nu}=0
\end{equation}
where the symmetric Einstein tensor $G^{\mu\nu}$ is defined as
\begin{equation}
G^{\mu\nu}\equiv R^{\mu\nu}-\frac{1}{2}g^{\mu\nu}R
\end{equation}
The Einstein equations can be used to express the covariant theory of gravity:
\begin{equation}
G_{\mu\nu}=R_{\mu\nu}-\frac{1}{2}g_{\mu\nu}R=\frac{8\pi G}{c^4}T_{\mu\nu}
\end{equation}
or, in a more compact form, in $c=G=1$ units
\begin{equation}
G_{\mu\nu}=8\pi T_{\mu\nu}
\end{equation}
Another way to express the Einstein equation is the alternative form
\begin{equation}
R_{\mu\nu}=\frac{8\pi G}{c^4}\left(T_{\mu\nu}-\frac{1}{2}g_{\mu\nu}T^\alpha_\alpha\right)
\end{equation}
\newpage
%-----------------------------------------------------------------------------------------------------
%-----------------------------------------------------------------------------------------------------
%-----------------------------------------------------------------------------------------------------
%-----------------------------------------------------------------------------------------------------
%-----------------------------------------------------------------------------------------------------
%-----------------------------------------------------------------------------------------------------
\section{Fractional Calculus}
Let us provide a brief introduction to several of the most common fractional calculus integrals and derivatives that will be utilized later on
\subsection{Fractional Integrals}
\begin{definition}
The left Riemann-Liouville fractional integral and the right Riemann-Liouville fractional integral are defined, respectively, by
\begin{equation}
\begin{aligned}
_aI^\alpha f(x,t)&=\frac{1}{\Gamma(\alpha)}\int_a^x(x-\xi)^{\alpha-1}f(\xi)\,d\xi\\
I_b^\alpha f(x,t)&=\frac{1}{\Gamma(\alpha)}\int_x^b(\xi-x)^{\alpha-1}f(\xi)\,d\xi
\end{aligned}
\end{equation}
where $\Gamma(\alpha)$ represents the gamma function.
\end{definition}
\begin{definition}
The left and right Riemann-Liouville fractional integrals of order $\alpha$ are respectivelly
\begin{equation}
_aI_x^\alpha f(x,t)=\frac{1}{\Gamma(\alpha)}\int_a^x(x-\xi)^{\alpha-1}f(\xi)\,d\xi
\end{equation}
and
\begin{equation}
_xI_b^\alpha f(x,t)=\frac{1}{\Gamma(\alpha)}\int_x^b(\xi-x)^{\alpha-1}f(\xi)\,d\xi
\end{equation}
\end{definition}
The Riesz fractional integral is given by
\begin{equation}
^R_aI_b^\alpha f(x,t)=\frac{1}{2}\left(_aI_x^\alpha f(x,t)+\; _xI_b^\alpha f(x,t)\right)
\end{equation}
\subsection{Fractional Derivatives}
\begin{definition}
The left Riemann-Liouville fractional derivative and the right Riemann-Liouville fractional derivatives are defined, respectively, by
\begin{equation}
\begin{aligned}
_aD^\alpha f(x)&=\frac{1}{\Gamma(n-\alpha)}\left(\frac{d}{dx}\right)^n\int_a^x(x-\xi)^{n-\alpha-1}f(\xi)\,d\xi\\
D_b^\alpha f(x)&=\frac{1}{\Gamma(n-\alpha)}\left(-\frac{d}{dx}\right)^n\int_x^b(\xi-x)^{n-\alpha-1}f(\xi)\,d\xi\\
\end{aligned}
\end{equation}
where $\alpha>0$ and $n=[\alpha]+1$, with $[\alpha]$ denoting the integer part of $\alpha$.
\end{definition}	
\begin{definition}
A couple carachteristics of the  Riemann-Liouville fractional derivatives
\begin{equation}
\begin{aligned}
_aD^0 f(x)&=D_b^0 f(x)=f(x)\\
_aD^n f(x)&=f^{(n)}(x)\\
D_b^n f(x)&=(-1)^nf^{(n)}(x)
\end{aligned}
\end{equation}
\end{definition}	
\begin{definition}
The left Caputo fractional derivative is
\begin{equation}
^C_aD^\alpha f(x)=\frac{1}{\Gamma(n-\alpha)}\int_a^x(x-\xi)^{n-\alpha-1}\left(\frac{d}{d\xi}\right)^nf(\xi)\,d\xi
\end{equation}
and the right Caputo derivative is
\begin{equation}
^CD_b^\alpha f(x)=\frac{1}{\Gamma(n-\alpha)}\int_x^b(\xi-x)^{n-\alpha-1}\left(-\frac{d}{d\xi}\right)^nf(\xi)\,d\xi
\end{equation}
(\textbf{Note:} by definition the Caputo fractional derivative of a constant is zero.)
\end{definition}
\begin{definition}
The Riemann-Liouville fractional derivatives and Caputo fractional derivatives are connected with each other by the following relations:
\begin{equation}
\begin{aligned}
^C_aD^\alpha f(x)&=\prescript{}{a}D^\alpha f(x)-\sum_{k=0}^{n-1}\frac{f^{(k)}(a)}{\Gamma(k-\alpha+1)}(x-a)^{k-\alpha}\\
^CD_b^\alpha f(x)&=D_b^\alpha f(x)-\sum_{k=0}^{n-1}\frac{(-1)^kf^{(k)}(b)}{\Gamma(k-\alpha+1)}(b-x)^{k-\alpha}
\end{aligned}
\end{equation}
\end{definition}
\begin{definition}
the Riesz fractional derivative is represented as
\begin{align}
^R_aD_b^\alpha f(x)=\frac{1}{2\cos(\pi\alpha/2)}\left(_aD^\alpha+D_b^\alpha\right)f(x)
\end{align}
\end{definition}
\begin{Lemma}
For a function $f(x)$ defined on the infinite domain $]-\infty,\infty[$, the following equality holds
\begin{equation}
\label{eq:lap}
-(-\Delta)^{\alpha/2}=-\frac{1}{2\cos(\pi\alpha/2)}\left(_{-\infty}D^\alpha f(x)+D_\infty^\alpha f(x)\right)
\end{equation}
where $(-\Delta)^{\alpha/2}$ is the laplace operator.
\end{Lemma}
\noindent To succinctly summarize the most common fractional derivatives, we will present them in tabular form:
\begin{table}[H]
\centering
\begin{NiceTabular}{c|l}
\toprule
\Block[fill=gray!10]{2-1}{Riemann-Liouville} & $_aD^\alpha f(x)=\frac{1}{\Gamma(n-\alpha)}\left(\frac{d}{dx}\right)^n\int_a^x(x-\xi)^{n-\alpha-1}f(\xi)\,d\xi $\\
 & $D_b^\alpha f(x)=\frac{1}{\Gamma(n-\alpha)}\left(-\frac{d}{dx}\right)^n\int_x^b(\xi-x)^{n-\alpha-1}f(\xi)\,d\xi$\\ \midrule
 \Block[fill=gray!10]{2-1}{Caputo} & $^C_aD^\alpha f(x)=\frac{1}{\Gamma(n-\alpha)}\int_a^x(x-\xi)^{n-\alpha-1}\left(\frac{d}{d\xi}\right)^nf(\xi)\,d\xi$\\
  & $^CD_b^\alpha f(x)=\frac{1}{\Gamma(n-\alpha)}\int_x^b(\xi-x)^{n-\alpha-1}\left(-\frac{d}{d\xi}\right)^nf(\xi)\,d\xi$\\ \midrule
\Block[fill=gray!10]{1-1}{Riesz} & $^R_aD_b^\alpha f(x)=\frac{1}{2\cos(\pi\alpha/2)}\left(_aD^\alpha+D_b^\alpha\right)f(x)$\\ \bottomrule
\end{NiceTabular}
\end{table}
\section{Gravitational Waves}
\subsection{Linearized Gravity}
The Einstein equation can be expressed in the equivalent form
\begin{equation}
\label{eq:einstein}
R_\mn=8\pi G\left(T_\mn-\frac{1}{2}g_\mn T_\lambda^\lambda\right)
\end{equation}
Due to the nonlinear nature of this equation, gravitational waves themselves act as a source of spacetime curvature, making it challenging to obtain wave solutions in the general case. However, in many situations, gravitational waves can be approximated as weak perturbations of the spacetime geometry, allowing the metric to be expressed in the form
\begin{equation}
\label{eq:perturbation}
g_\mn(x)=\eta_\mn+h_\mn(x)
\end{equation}
where $\eta_\mn$ represents the metric of flat Minkowski space and $h_\mn$ is a small perturbation. The linearized vacuum Einstein equation is then derived by substituting \eqref{eq:perturbation} into the vacuum Einstein equation
\begin{equation}
\label{eq:vacuum}
R_\mn=0
\end{equation}
obtained from \eqref{eq:einstein} by setting the stree-energy tensor $T_\mn$ to zero, and expanding the resulting equations to first order in $h_\mn$.
\newpage
\subsection{Linearized Curvature Tensor}
The Ricci curvature tensor $R_\mn$ appearing in \eqref{eq:vacuum} is given by 
\begin{equation}
\label{eq:Ricci}
R_\mn=\Gamma^\gamma_{\mn,\gamma}-\Gamma^\gamma_{\mu\gamma,\nu}+\Gamma^\gamma_{\mn}\Gamma^\sigma_{\gamma\sigma}-\Gamma^\sigma_{\mu\gamma}\Gamma^\gamma_{\nu\sigma}
\end{equation}
where the Cristoffel symbols $\Gamma^\gamma_\mn$ are related to the metric tensor $g_\mn$ by
\begin{equation}
\Gamma^\gamma_\mn=\frac{1}{2}g^{\gamma\delta}\left(\frac{\partial g_{\nu\delta}}{\partial x^\mu}+\frac{\partial g_{\mu\delta}}{\partial x^\nu}-\frac{\partial g_\mn}{\partial x^\delta}\right)
\end{equation}
To zeroth order in $h_\mn$ the Christoffel coefficients vanish and so does $R_\mn$. To the first order in $h_\mn$
\begin{equation}
\label{eq:dG}
\delta \Gamma^\gamma_\mn=\frac{1}{2}\eta^{\gamma\delta}\left(\frac{\partial h_{\nu\delta}}{\partial x^\mu}+\frac{\partial h_{\mu\delta}}{\partial x^\nu}-\frac{\partial h_\mn}{\partial x^\delta}\right)
\end{equation}
The last two termn in \eqref{eq:Ricci} are quadratic in $h$ and can therefore be discarded, resulting in
\begin{equation}
\label{eq:dR}
\delta R_\mn=\frac{\partial(\delta\Gamma^\lambda_\mn)}{\partial x^\lambda}-\frac{\partial(\delta\Gamma^\gamma_{\mu\lambda})}{\partial x^\nu}+\mathcal{O}(h^2)
\end{equation}
to first order in $h$.
\subsection{Wave equation}
Substitution of \eqref{eq:dG} in \eqref{eq:dR} results in
\begin{equation}
\delta R_\mn=\frac{1}{2}(-\Box h_\mn+\partial_\mu V_\nu+\partial_\nu V_\mu)
\end{equation}
where the 4-dimensional Laplacian operator (d'Alembertian operator) is defined as
\begin{equation*}
\Box=-\frac{\partial^2}{\partial t^2}+\nabla^2
\end{equation*}
with $\partial_\mu=\partial/\partial x^\mu$, and the $V_\nu$ are defined through
\begin{equation}
\begin{aligned}
V_\nu&=\partial_\lambda h_\nu^\lambda-\frac{1}{2}\partial_\nu h_\lambda^\lambda\\
&=\partial_\lambda \eta^{\lambda\delta}h_{\delta\nu}-\frac{1}{2}\partial_\nu \eta^{\lambda\delta}h_{\delta\lambda}
\end{aligned}
\end{equation}
Thus, the vacuum Einstein equation yields
\begin{equation}
\label{eq:vac}
\Box h_\mn-\partial_\mu V_\nu-\partial_\nu V_\mu=0
\end{equation}
\subsection{Gauge Transformations}
Keeping in mind gauge invariance, we may apply certain gauge transformations to solve the last equation in the previous section, which will preserve $\eta_\mn$ and simply modify the form of $h_\mn$.
\begin{equation}
x^\mu\to x^{'\mu}=x^\mu+\epsilon^\mu(x)
\end{equation}
where $\epsilon^\mu(x)$ is similar in size to $h_\mn$, the metric is then changed
\begin{equation}
\begin{aligned}
g_\mn(x)&=\eta_\mn+h_\mn(x)\to\eta_\mn+h^{'}_\mn(x)\\
&=\eta_\mn+h_\mn(x)-\partial_\mu\epsilon_\nu-\partial_\nu\epsilon_\mu
\end{aligned}
\end{equation}
The transformation $h_\mn \to h^{'}_\mn$ is
\begin{equation}
h^{'}_\mn=h_\mn-\partial_\mu\epsilon_\nu-\partial_\nu\epsilon_\mu
\end{equation}
is called a gauge transformation (If $h_\mn$ is a solution to \eqref{eq:vac} than so is $h^{'}_\mn$)\\
\\
Adopting a standard gauge choice, akin to selecting the Lorentz gauge in electromagnetism, allows for the linearized vacuum gravitational equations to be replaced by the following two equations:
\begin{align}
\Box \overline h_\mn(x)&=\left(-\frac{\partial^2}{\partial t^2}+\nabla^2\right)\overline h_\mn=0 \label{eq:waveqn}\\[.5ex]
&\partial_\nu\overline h^\mn(x)=0 \label{eq:gauge}
\end{align}
where \eqref{eq:waveqn} is a wave equation corresponding to the linearized Einstein equation, \eqref{eq:gauge} is a Lorentz gauge constraint, and the trace-reversed amplitude is defined by
\begin{equation}
\label{eq:trac}
\overline h_\mn=h_\mn-\frac{1}{2}\eta_\mn h
\end{equation}
where $h=h_\alpha^\alpha$ is the trace.
\subsection{Weak Gravitational Waves}
Solutions of \eqref{eq:waveqn} and \eqref{eq:gauge} are expected to take the form of a superposition of plane-wave components, expressed as
\begin{equation}
\overline h_\mn(x)=\alpha_\mn e^{ik\cdot x}=\alpha_\mn e^{ik_\alpha x^\alpha}
\end{equation}
where the polarization tensor $a_\mn$ can be represented by a constant $4\times 4$ matrix and $k$ denotes the 4-wavevector ($k^\mu=(\omega/c,\bm k$)), where $\omega$ represents the angular frequency of the wave.\\
To simplify the mathematical analysis, let us consider a gravitational wave propagating solely in the $x$-direction.
\begin{equation}
\label{eq:z}
\overline h_\mn(x)=\overline h_\mn(t,x)=\alpha_\mn e^{i(kx-\omega t)}
\end{equation}
The linearized Einstein equation then simply becomes
\begin{equation}
\Box \overline h_\mn(t,z)=\left(-\frac{\partial^2}{\partial t^2}+\frac{\partial^2}{\partial x^2}\right)\overline h_\mn=0
\end{equation}
\subsection{A Factional Calculus Approach}
\subsubsection{First method}
Utilizing the previously defined Laplacian operator \eqref{eq:lap} he d'Alembertian operator simplifies to (accounting solely for fractional calculus in the spatial part to simplify the analysis):
\begin{equation}
\Box_\alpha f=-\frac{\partial^2f}{\partial t^2}+\frac{1}{2\cos(\pi\alpha/2)}\left(_{-\infty}D^\alpha f+D_\infty^\alpha f\right)
\end{equation}
Let's explore whether $h_\mn=\alpha_\mn e^{i(kx-\omega t)}$ still constitutes a type of solution for the wave equation. Given that in the preceding section, the $x$-direction was chosen, the linearized Einstein equation simplifies to
\begin{equation*}
\begin{aligned}
\Box_\alpha \overline h_\mn(t,x)&=\left(-\frac{\partial^2 h_\mn}{\partial t^2}+\frac{1}{2\cos(\pi\alpha/2)}\left(_{-\infty}D^\alpha h_\mn+D_\infty^\alpha h_\mn\right)\right)=0\\
2\cos(\pi\alpha/2)\frac{\partial^2 h_\mn}{\partial t^2}&=\frac{1}{\Gamma(n-\alpha)}\left(\frac{d}{dx}\right)^n\int_{-\infty}^x(x-\xi)^{n-\alpha-1}h_\mn(t,\xi)\,d\xi\\
&+\frac{1}{\Gamma(n-\alpha)}\left(-\frac{d}{dx}\right)^n\int_x^\infty(\xi-x)^{n-\alpha-1}h_\mn(t,\xi)\,d\xi
\end{aligned}
\end{equation*}
if we choose $1<\alpha<2$, then $n=2$ and this simplifies a lot
\begin{align*}
2\cos(\pi\alpha/2)\frac{\partial^2 h_\mn}{\partial t^2}&=\frac{1}{\Gamma(2-\alpha)}\frac{d^2}{dx^2}\Biggl[\int_{-\infty}^x(x-\xi)^{1-\alpha}h_\mn(t,\xi)\,d\xi\\
&+\int_x^\infty(\xi-x)^{1-\alpha}h_\mn(t,\xi)\,d\xi\Biggr]\\
2\cos(\pi\alpha/2)\frac{\partial^2 h_\mn}{\partial t^2}&=\frac{1}{\Gamma(2-\alpha)}\frac{d^2}{dx^2}\Biggl[\int_{-\infty}^x(x-\xi)^{1-\alpha}\alpha_\mn e^{i(k\xi-\omega t)}\,d\xi\\
&+\int_x^\infty(\xi-x)^{1-\alpha}\alpha_\mn e^{i(k\xi-\omega t)}\,d\xi\Biggr]\\
2\cos(\pi\alpha/2)\frac{\partial^2 h_\mn}{\partial t^2}&=\frac{\alpha_\mn e^{-i\omega t}}{\Gamma(2-\alpha)}\frac{d^2}{dx^2}\Biggl[\int_{-\infty}^x(x-\xi)^{1-\alpha}e^{ik\xi}\,d\xi\\
&+\int_x^\infty(\xi-x)^{1-\alpha}e^{ik\xi}\,d\xi\Biggr]
\end{align*}
However, upon reaching this final equation, manual solving becomes exceedingly challenging. As of today, only a limited number of programs possess the capability to compute derivatives of fractional order, and even these are confined to a select few functions. Consequently, it remains uncertain whether $h_\mn=\alpha_\mn e^{i(kx-\omega t)}$ would still be a solution within the realm of fractional calculus.
\subsubsection{Second method}
Let's consider the scenario where only the right-side Caputo derivatives are utilized, as these can be calculated using available programs. Additionally, we will also take fractional calculus into account for the time component.
\begin{align*}
&\frac{1}{\Gamma(n-\alpha)}\int_0^\infty(\xi-x)^{n-\alpha-1}\left(-\frac{d}{d\xi}\right)^nf(\xi,x)\,d\xi=\\&\hspace{2cm}=\frac{1}{\Gamma(n-\alpha)}\int_0^\infty(\xi-x)^{n-\alpha-1}\left(-\frac{d}{d\xi}\right)^nf(t,\xi)\,d\xi
\end{align*}
Once again let's take $1<\alpha<2$ to simplify things
\begin{align*}
\int_0^\infty(\xi-x)^{1-\alpha}\frac{d^2}{d\xi^2}f(\xi,x)\,d\xi=\int_0^\infty(\xi-x)^{1-\alpha}\frac{d^2}{d\xi^2}f(t,\xi)\,d\xi
\end{align*}
Let's revisit whether $h_\mn=\alpha_\mn e^{i(kx-\omega t)}$ satisfies this equation. This particular Caputo derivative can be computed using Mathematica. Given that one of the conditions stipulates $1<\alpha<2$ let's select $\alpha=3/2$, the result will then be
\begin{equation}
\label{eq:sol}
\begin{fleqn}
\begin{aligned}
&\alpha_\mn\left(\frac{e^{ikx}}{2\sqrt \pi t^{3/2}}+\frac{ie^{ikx}\omega}{\sqrt \pi \sqrt t}+\frac{e^{i(kx-\omega t)}(-it\omega)^{3/2}(1-Q(-3/2,-i\omega t))}{t^{3/2}}\right)=\\
&=\alpha_\mn\left(\frac{e^{-i\omega t}}{2\sqrt \pi x^{3/2}}+\frac{ie^{-i\omega t}k}{\sqrt \pi \sqrt x}+\frac{e^{i(kx-\omega t)}(ikx)^{3/2}(1-Q(-3/2,ikx))}{x^{3/2}}\right)
\end{aligned}
\end{fleqn}
\end{equation}
where $Q(\alpha,x)$ is the regularized gamma function
\begin{equation*}
Q(\alpha,x)=\frac{\Gamma(\alpha,x)}{\Gamma(\alpha)}
\end{equation*}
Thus, $h_\mn=\alpha_\mn e^{i(kx-\omega t)}$ constitutes a solution to the gravitational wave equation if and only if \eqref{eq:sol} is satisfied, a task that is also quite challenging to perform manually.
\section{Conclusions}
Unfortunately, there isn't much more that can be done than this because the equations that fractional calculus produces are too complicated to solve by hand. This is all I could do because there isn't any software or programming language available right now that can calculate these kind of equations.
\newpage
\nocite{*}
\bibliographystyle{plain}
\bibliography{GWFC}
\end{document}