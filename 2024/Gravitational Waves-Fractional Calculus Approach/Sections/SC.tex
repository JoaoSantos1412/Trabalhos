As seen in \eqref{eqn:STC},
\begin{equation}
x\equiv x^\mu=(x^0,x^1,x^2,x^3)=(ct,\bm x)
\end{equation}
where $\bm x$ denotes the spacial components $(x^1,x^2,x^3)$. The first coordinate ($x^0$) is called timelike and the three last components $(\bm x)$ are called spacelike. The placement of the $\mu$ index is also important as there will be a difference in the sign of the 4-vector. (\emph{Note: the bold vector $\bm x$ is a normal vector that we are used to, whereas the non-bold “vector” $x$ is the 4-vector.})