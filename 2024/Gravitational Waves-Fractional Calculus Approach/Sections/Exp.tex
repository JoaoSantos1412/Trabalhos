\subsubsection{Expansion of vectors and dual vectors}
We can express an arbitrary vector $\bm V$ in terms of the tangent basis $(\bm e_i)$, and similarly, an arbitrary dual vector $\bm \omega$ in terms of the dual basis $(\bm e^i)$:
\begin{align}
\label{eqn:v}
&\bm V = V^1\bm e_1 + V^2\bm e_2 + V^3\bm e_3=\sum_i V^i\bm e_i\equiv V^i\bm e_i,\\
&\bm \omega = \omega_1\bm e^1 + \omega_2\bm e^2 + \omega_3\bm e^3=\sum_i \omega_i\bm e^i\equiv \omega_i\bm e^i.
\end{align}
where the upper-index coefficients $V^i$ are referred to as the components of the vector in the basis $\bm e_i$ and the lower-index coefficients $\omega_i$ are referred to as the components of the dual vector in the basis $\bm e^i$. These components, as they are in different basis, usually are distinct, nevertheless, there is a relation between their spaces.