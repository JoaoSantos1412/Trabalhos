\subsubsection{Vector Scalar Product and the Metric Tensor}
The scalar product of two vector can be written as follows
\begin{equation}
\label{eqn:PE1}
\bm A \cdot \bm B=(A^i\bm e_i)\cdot(B^j\bm e_j)=\bm e_i \cdot\bm e_j A^iB^j=g_{ij}A^iB^j
\end{equation}
where $g_{ij}$ are the metric tensor components (in this basis) and are defined by
\begin{equation}
g_{ij}\equiv\bm e_i \cdot\bm e_j
\end{equation}
The same can be done to the scalar product of two dual vectors:
\begin{equation}
\label{eqn:PE2}
\bm\alpha\cdot\bm\beta=(\alpha_i\bm e^i)\cdot(\beta_j\bm e^j)=\bm e^i \cdot\bm e^j\alpha_i\beta_j=g^{ij}\alpha_i\beta_j
\end{equation}
where the metric tensor components (in this basis) $g^{ij}$ are defined by
\begin{equation}
g^{ij}\equiv\bm e^i \cdot\bm e^j
\end{equation}
It is also possible to have the scalar product between a dual vector and a vector (and vice versa)
\begin{equation}
\label{eqn:PE3}
\bm\alpha\cdot\bm B=(\alpha_i\bm e^i)\cdot(B^j\bm e_j)=\bm e^i \cdot\bm e_j \alpha_iB^j=g^i_j\alpha_iB^j
\end{equation}
where the metric tensor $g^i_j$ is defined by
\begin{equation}
g^i_j\equiv\bm e^i \cdot\bm e_j
\end{equation}
The basis $\bm e^i$ and $\bm e_i$ are defined in such a way that:
\begin{equation}
\bm e^i\cdot\bm e_j=\delta^i_j=
\begin{cases}
1 \text{ if } i=j\\
0 \text { if } i\neq j
\end{cases}
\end{equation}