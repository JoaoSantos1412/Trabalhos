Let us introduce two pivotal concepts within the domain of general relativity: the notation conventions and the metric utilized throughout this paper.
\subsection{Spacetime Notation}
Minkowski, in order to simplify special relativity, united the three dimensions of space and the one dimension of time into the so called 4-dimensional spacetime, whose coordinates are denoted as:
\begin{equation}
\label{eqn:STC}
(x^0,x^1,x^2,x^3)\equiv(ct,x,y,z)
\end{equation}
where the superscripts are indices (and not exponents!). The time component $(ct)$ is adjusted by the speed of light to ensure consistency in units with the other coordinates. This is also commonly written in $c=1$ units, and therefore
\begin{equation}
(x^0,x^1,x^2,x^3)\equiv(t,x,y,z)
\end{equation}
\subsection{Metric Tensor}
\begin{equation}
\begin{aligned}
ds^2=\sum_{\mu\nu}\eta_{\mu\nu}dx^\mu dx^\nu&=-(dx^0)^2+(dx^1)^2+(dx^2)^2+(dx^3)^2,\\
&=-c^2dt^2+dx^2+dy^2+dz^2.
\end{aligned}
\end{equation}
The quantity $\eta_\mn$ is represented by the following diagonal matrix
\begin{equation}
\label{eqn:diag}
\eta_{\mu\nu}=
\begin{pmatrix}
-1 & 0 & 0 & 0\\
0 & 1 &0 & 0\\
0 & 0 & 1 & 0\\
0 & 0 & 0 & 1
\end{pmatrix}
=\diag{-1,1,1,1}
\end{equation}
and is referred to as the metric tensor of Minkowski space. (a lot of books choose the symmetric metric $\eta_\mn=\diag{1,-1,-1,-1}$, however the one mentioned in eq.  \eqref{eqn:diag} is the most used in relativity.)