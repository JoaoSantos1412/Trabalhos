\subsubsection{Coordinate and Non-coordinate Bases}
The basis vectors can be written as
\begin{equation}
e_\mu=\frac{\partial}{\partial x^\mu}=\partial_\mu
\end{equation}
wich then allows an arbitrary vector to be expanded as
\begin{equation}
V=V^\mu e_\mu=V^\mu\partial_\mu
\end{equation}
The distance between two nearby points is $ds=e_\mu(x) dx^\mu$, thus
\begin{equation}
ds^2=ds\cdot ds=(e_\mu\cdot e_\nu)dx^\mu dx^\nu=g_\mn dx^\mu dx^\nu
\end{equation}
where the metric tensor $g_\mn$ is defined by
\begin{equation}
g_\mn(x)=e_\mu(x)\cdot e_\nu(x)
\end{equation}
The scalar product of two vector can then be written as
\begin{equation}
A\cdot B=(A^\mu e_\mu)\cdot(B^\nu e_\nu)=g_\mn A^\mu B^\nu
\end{equation}
Likewise, a set of dual basis vectors ($e^\mu$) may be used to expand dual vectors $\omega$ as
\begin{equation}
\omega=\omega_\mu e^\mu
\end{equation}
and the scalar of two dual vectors
\begin{equation}
\alpha\cdot\beta=g^\mn\alpha_\mu\beta_\nu
\end{equation}
we can also calculate the scalar product between a vector and a dual vector
\begin{equation}
\omega(V)=\omega_\mu e^\mu(V^\nu e_\nu)=g_\nu^\mu\omega_\mu V^\nu =\omega_\mu V^\mu
\end{equation}
where
\begin{equation}
g^\nu_\mu=\delta_\nu^\mu=
\begin{cases}
1\quad\mu=\nu\\
0\quad\mu\neq\nu
\end{cases}
\end{equation}